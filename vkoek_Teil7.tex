\begin{appendix}

\chapter{Appendix}


%########################################
\section{\label{sec:App-quadratwurzel}Zu Abschnitt \protect\ref{sec:zufall}: 
``Quadratwurzelgesetz'' im Europaparlament}
%########################################


Vereinfacht geht die Abstimmung im Europaparlament zweistufig
vonstatten:
\bi
\item Die Vertreter jedes der 25 L\"ander (Einwohnerzahl $N_i$, i=1,
\ldots,  25) stimmen mit ``Nein'' ($Y_i=0$) oder ``Ja'' ($Y_i=1$) ab; 
die letztendliche Entscheidung $Z$ im Parlament h\"angt nach folgender
Regel von den L\"andergewichten $w_i$ und dem Quorum $R$ ab:
\bdm
Z=\twoCases{1 \text{(``Ja'')}}{\text{falls} \ \sum\limits_{i=1}^{25}w_iY_i\ge R,}
      {0 \text{(``Nein'')}}{\text{sonst.}}
\edm

\item Die Vertreter selbst werden direkt (Referendum) oder indirekt
(\"uber die Parteien) vom Volk gew\"ahlt.
\ei
die Bef\"urworter der ``Quadratwurzelregel'' schlagen folgende Formel f\"ur die
L\"andergewichte vor und behaupten, dass sie im
statistischen Mittel f\"ur gr\"o\3tm\"ogliche
Gleichheit der Wahl-Einflussm\"oglichkeiten (``Voting power'') der
einzelnen B\"urger sorgt:
\bdm
w_i=\frac{\sqrt{N_i}}{\sum\limits_{j=1}^{25} \sqrt{N_j}}.
\edm
Die ``Voting Power'' ist dabei  gleich der Wahrscheinlichkeit daf\"ur, dass
eine individuelle Stimme bei der Wahlentscheidung genau die entscheidende
ist. 

%#############################################
\aufgabenbox{\Einstein Aufgabe zum Quadratwurzelgesetz}{
%#############################################
Pr\"ufen Sie unter Verwendung von \refkl{VzufallBin}, den
Definitionen von Verteilungs- und Dichtefunktionen (siehe Statistik I)
sowie dem Zentralen
Grenzwertsatz (Statistik II), ob die ``Quadratwurzelregel'' zu einer
gleichen ``Voting Power'' f\"ur alle  B\"urger zumindest unter folgenden
Annahmen f\"uhrt:
\bi
\item Alle B\"urger und
alle L\"ander entscheiden unabh\"angig voneinander.
\item In jedem
Land gibt es nur zwei relevante Parteien oder Koalitionen.
\item Die mittlere Zustimmungsquote in allen L\"andern  liegt bei 0.5.
\item Die Voting
Power eines Landes (also die Wahrscheinlichkeit daf\"ur, dass gerade
die Stimme $Y_i$ dieses Landes wahlentscheidend ist) w\"achst proportional mit
dem L\"andergewicht $w_i$.
\ei
}

%#############################
\bfsf{L\"osung}
%#############################
Die Stimme $Y_i$ des  Landes $i$ (mit den Werten 1 f\"ur ``Ja'' und 0
f\"ur ``Nein'') h\"angt nach der
Aufgabenstellung  (bei 100\% Wahlbeteiligung) von der Zustimmungsquote
$Y_i$
der B\"urger ab,
\bdm
Y_i=\frac{1}{N_i}\sum\limits_{j=1}^{N_i} Y_{ij},
\edm
und damit letztendlich von jeder einzelnen Stimme $Y_{ij}$ der
B\"urger in diesem Land. Bei zwei Parteien/Koalitionen
(Mehrheitsentscheidung) ist der
Erwartungswert der Landes-Stimme gleich der Wahrscheinlichkeit
daf\"ur, dass die Zustimmungsquote oberhalb 50\% liegt:
\bdm
E(Y_i)=P(Y_i>0.5)
\edm
Die Voting Power $P_{ij}=E(Y_i|Y_{ij}=1)-E(Y_i|Y_{ij}=0)$ der Stimme
$Y_{ij}$ eines B\"urgers $j$ bez\"uglich der
Landes-Entscheidung $Y_i$ ist
offensichtlich durch die \"Anderung der bedingten Wahrscheinlichkeiten
gegeben:
\be
\label{Pij}
P_{ij}=P(Y_i>0.5 | Y_{ij}=1) - P(Y_i>0.5 | Y_{ij}=0).
\ee
Da die Stimmen der anderen B\"urger $j'\neq j$ nach Voraussetzung
unabh\"angig von $j$ sind, kann man auch schreiben
\bdm
P_{ij}=P\left(\frac{1}{N_i}\sum\limits_{j'} Y_{ij'}>0.5-\frac{1}{N_i}\right)
      - P\left(\frac{1}{N_i}\sum\limits_{j'} Y_{ij'}>0.5\right).
\edm
Mit der allgemeinen Definition von Verteilungsfunktion (Vgl. Statistik-Vorlesung)
$F_y(y)=P(Y\le y)$ und ihrer Dichte $f_y(y)=\abl{F(y)}{y}$ erh\"alt
man
\be
\label{Pijf}
P_{ij}=F_{y_i}(0.5)-F_{y_i}(0.5-1/N_i)\approx
\frac{f_{y_i}(0.5)}{N_i}.
\ee

Die Zustimmungsquote $Y_i$ der B\"urger eines Landes und damit ihre Dichtefunktion
$f_{y_i}(y)$ gehorchen der selben statistischen Verteilung  wie die Anteilswerte in Abschnitt
\ref{sec:zufallAnteil}, also $E(Y_i)=\mu$ und
$V(Y_i)=\mu(1-\mu)/N_i$. Wegen der gro\3en Bev\"olkerungszahl gilt auf
jedem Fall der zentrale Grenzwertsatz, $Y_i$ ist also normalverteilt
mit Mittelwert $\mu$ und Varianz $\mu(1-\mu)/N_i$. Setzt man die
Dichtefunktion dieser Normalverteilung in Gl. \refkl{Pijf} ein, erh\"alt man
\be
\label{Pijgauss}
P_{ij}= \sqrt{\frac{1}{2 N_i \pi\mu(1-\mu)}}
e^{-\frac{N_i(0.5-\mu)^2}{2\mu(1-\mu)}}.
\ee
Da nun nach Voraussetzung die Voting Power $Q_i=E(Z|Y_i=1)-E(Z|Y_i=0)$ eines Landes bei der
Europawahl proportional seiner Gewichtung $w_i$ ist, $Q_i=\alpha w_i$,
gilt f\"ur die Voting-Power $Q_{ij}$ eines
Einzelb\"urgers bez\"uglich der Europarat-Entscheidung nach \refkl{Pij}
\be
\label{Qij}
Q_{ij}=E(Z|Y_{ij}=1)-E(Z|Y_{ij}=0) = \alpha w_i P_{ij}.
\ee
Setzt man nun gem\"a\3 Aufgabenstellung 
 in allen L\"andern denselben Mittelwert $\mu=0.5$ f\"ur die
Zustimmungsquote voraus, erh\"alt man nach Einsetzen von
\refkl{Pijgauss} in \refkl{Qij}
\bdm
Q_{ij}=\alpha w_i  \sqrt{\frac{2}{N_i \pi}}.
\edm
Damit haben in Europawahlen alle B\"urger denselben Wert der
Voting Power, falls 
\bdm
w_i \propto \sqrt{N_i},
\edm
womit der ``Quadratwurzelvorschlag''
begr\"undet ist. 

\textit{Bemerkung}W\"ahrend das Ergebnis \refkl{Pijgauss} von den
Arbeit von Penrose und Banzhaff schon lange bekannt ist, wurde die
Anwendung auf die Europawahl mit den zus\"atzlichen Annahmen
\refkl{Qij} und $\mu=0.5$ erst k\"urzlich vorgestellt.\footnote{Zwei
der vier Autoren der wichtigsten Arbeit zu diesem Thema sind Polen.}
Die Annahme \refkl{Qij} ist durchaus nichttrivial und nur f\"ur
bestimmte Werte des Quorums (im konkreten Fall $R=0.62$) hinreichend gut erf\"ullt.


%########################################
\section{\label{sec:App-opt}Zu Abschnitt \protect\ref{sec:stichpr}: 
Optimale Aufteilung der Schichten einer Stichprobe}
%########################################
\EinsteinBeg 

Offensichtlich kann man mit Entzerrungsfaktoren auch eine Stichprobe mit
``falschen'' Anteilen der verschiedenen Klassen des Quotenmerkmals
erwartungstreu sch\"atzen. Dies erm\"oglicht es, eine 
Quoten-Stichprobe  mit gezielt verzerrten Quoten zu ziehen unter der
Zielsetzung das die Varianz  \refkl{V-entzerr} des entzerrenden Erwartungswertsch\"atzers
$\hat{\mu}\sup{(E)}$ minimal wird, vgl. Abb. \ref{fig:entzerr}. Konkret wird
bei festem Stichprobenumfang $n=\sum_k n_k$ die Varianz
\refkl{V-entzerr} nur in Abh\"angigkeit der festen wahren Anteile und
der  einzelnen
absoluten Stichprobenh\"aufigkeiten $n_k$ ausgedr\"uckt
(vgl. zweite Zeile der Herleitung dieser Formel) und unter der
Nebenbedingung eines festen Stichprobenumfangs minimiert:
\be
V(\hat{\mu}\sup{(E)}) =  V(n_1,n_2, \ldots)=\sum\limits_k \theta_k^2 \frac{\sigma_k^2}{n_k}
\stackrel{!}{=} \min\limits_{n_1,n_2,...}, \quad \sum_k n_k=n.
\ee

Dies geschieht mit dem sog. ``Lagrange-Verfahren'', bei der nicht die
Funktion selbst, sondern die Summe aus der Funktion und der mit sog.
``Lagrange-Faktoren''  multiplizierten Nebenbedingungen gleich 0 gesetzt wird:
\bdm
\ablpart{V(n_1,n_2, \ldots)}{n_k} 
+ \lambda
\ablpart{}{n_k}\left(\sum\limits_{k'}n_{k'}-n\right)\stackrel{!}{=}0.
\edm
Dies f\"uhrt zu
\bdm
\frac{-\theta_k^2\sigma_k^2}{n_k^2} + \lambda=0,
\edm
also ist f\"ur alle $k$ das Produkt $\theta_k\sigma_k/n_k$ konstant,
also
$n_k$ proportional zu $\theta_k\sigma_k$ Aus der Nebenbedingung
$\sum_kn_k=n$ ergibt sich damit folgende ``optimale'' Schichtung,
ausgedr\"uckt als relativen
Stichprobenh\"aufigkeiten $f_k$ der Klassen des Schichtungsmerkmals:
\be
\label{fk-opt}
f_k\sup{opt}=\frac{n_k\sup{opt}}{n}
=\frac{\theta_k\sigma_k}{\sum\limits_{k'}\theta_{k'}\sigma_{k'}}
\ee
und nach Einsetzen in die Varianzformel die minimierte Varianz selbst:
\be
\label{V-opt}
V\sup{opt}(\hat{\mu}\sup{(E)})=\frac{1}{n}\left(\sum\limits_k \theta_k\sigma_k\right)^2.
\ee

\aufgabenbox{Aufgabe: Studentenstadt}{Berechnen Sie in dem bereits bei der Schichtung und
Entzerrung verwendeten Beispiel der ``Studentenstadt''  die optimale Schichtung bez\"uglich
Studenten und Nichtstudenten und die bei einem Stichprobenumfang von
2\ 500 resultierende Varianz und Standardabweichung des Merkmals
``MIV-Bevorzugung''.
}

\bfsf{L\"osung:}

\bdm
n_1\sup{opt}=843, \quad n_2\sup{opt}=n-n_1\sup{opt}=1\ 657
\edm

\bdm
E_1=1.186, \quad E_2=0.906, \quad V(\hat{\mu}\sup{(E)})=6.14*10^{-5},
\edm
und damit eine Standardabweichung von $\sqrt{V}=0.784\%$.
Dies ist nur minimal besser als die Entzerrung bei
Zufallsauswahl. Allgemein liefert die ``optimale Schichtung'' nur dann
nennenswerte Vorteile, wenn die Varianzen innerhalb der Schichten
sich stark unterscheiden. Dann werden die Schichten mit
vergleichsweise gro\3er
Varianz \"uberproportional ber\"ucksichtigt. 


\end{appendix}