%########################################################
% skeletons fuer latex2e documente
% AUSFUEHRLICHERE Version: skeletonAusfuehrlich.tex (dec04)
%########################################################
\documentclass[12pt,a4paper]{scrartcl}  
%\documentclass[twocolumn,showpacs,preprintnumbers,amsmath,amssymb]{revtex4}
%\documentclass[preprint,showpacs,preprintnumbers,amsmath,amssymb]{revtex4}
%\documentclass[a4paper]{foils}
     % Praesentation a la "Powerpoint", vgl. ~/prosperPraesentation/          
%\documentclass[tu,colorBG,slideColor,pdf]{prosper} 

\usepackage{graphicx} % definiert \includegraphics[width=???mmm]{eps Bild}
\usepackage{lscape} % provides landscape environment; OK for figs+text(mai05)
\usepackage{ngerman}  %''new german''; (nicht als Argument in \documentclass)
%\usepackage[ngerman]{babel}  %nicht klar, ob dies besser ist als ngerman, "vergisst" \3 Definition

\usepackage{eurosym}  %Euro-Symbol: \euro{Zahl} oder euro{}
\usepackage[latin1]{inputenc} % dann kann auch � direkt �bersetzt werden!
\usepackage[dvips]{color}     %Definiert \definecolor und 
                              % red,yellow,green,blue,black,gray,white

\usepackage{hyperref} 
%                  Internet-Link: \href{http://www.WasAuchImmer.html}
%                  {\blue{\underline{TextDesLinks} }}
%                  Lokaler Link: \hyperlink{targetName}{TextDesLinks}
%                  Link-Target: \hypertarget{targetName}
%                  Achtung: lokale Links funktionieren z Z nicht
%  Info: \myHyperlink{http..}{Linktext}

\usepackage{cite} % z.B. Refs 1,2,3,5 werden zu [1-3,5]  zusammengefasst

\input{defs}          % includes auch ``colors.st''

\usepackage{graphicx}
\usepackage[latin1]{inputenc}
\usepackage{ngerman}
\usepackage{psfrag}
\usepackage{umlaute}



%########################################################

\begin{document}
\title{Aufgabe zur Bemessungsverkehrsst\"arke}
%\author{Treibi}
\author{\large\textsf\textbf{Martin Treiber}}
\date{\normalsize\textsf\textbf{\today}}
%\date{\normalsize\textsf\textbf{32.01.2019}} % if ``timestamp'' fixed

\maketitle % sonst wird nix gemacht!

\subsection*{Aufgabe}
Ein das gesamte Jahr \"uber in Betrieb befindlicher Detektor hat
folgende Verkehrsbelastungen gemessen (alle Zahlenwerte in Fahrzeuge
pro Stunde):
\bi
\item an 150 Normalwerktagen (Dienstag-Donnerstag) von 0-6 h unter
200,  von 6-8 h zwischen 1000 und 1200, von 8h-18h 800-1000 und
18h-24h zwischen 300 und 500,
\item an insgesamt 100 Montagen und Freitagen von 0-6 h zwischen 200
und 300, 6-8h 1200-1400, 8-18h 800-1000 und ab 18 h 300-500
\item an insgesamt 100 Sams- und Sonntagen von 8h bis 16 h zwischen
500 und 800 und in der restlichen Zeit 300-500,
\item an 15 Spitzentagen w\"ahrend jeweils 5 h von
1400-1600, 8h von 1200-1400 und den Rest 500-800.
\ei

\bi
\item[(a)] Klassieren Sie die st\"undlichen Verkehrsaufkommen geeignet
und berechnen Sie f\"ur jede Klasse die relative H\"aufigkeit
(Zeitanteil) des Auftretens in den 365 gemessenen Tagen.
\item[(b)] Ermitteln Sie - wie in der Vorlesung Statistik I - die
 Verteilungs- und Dichtefunktion der Verkehrsbelastungen.
\item[(c)] Berechnen Sie das 90. Perzentil der Verteilung. Was sagt es
aus?
\item[(d)] F\"ur die sogenannte ``Bemessungsverkehrsst\"arke'' wird
die Belastung herangezogen, welche nur in 30 h bzw. 150 h pro Jahr (je
nach Definition) \"uberschritten wird. Welchem Quantilswerten
entsprechen diese zwei Bemessungsverkehrsst\"arlen und wie gro\3 sind
sie?
\item[(e)] Die Kapazit\"at einer geplanten neuen Strecke wird nun nach der
``150-Stunden-Bemessungsverkehrsst\"arke'' ausgelegt, d.h. in
j\"ahrlich 150
Stunden gibt es \"Uberlastungen und Staus. 
Die resultierende Zeitverz\"ogerung wird
vereinfacht durch folgende st\"uckweise lineare CR-Funktion abgesch\"atzt:
\bdm
\frac{T(Q)}{T_0}=\twoCases{1}{Q\le K}{1+5\left(\frac{Q}{K}-1\right)}{Q>K.}
\edm
Wie gro\3 ist die resultierende Zeitverz\"ogerung durch Staus pro Jahr
und Kilometer einer derart bemessenen Strecke, wenn das Tempo auf der
unbelasteten Strecke 60 km/h betr\"agt und eine Verteilung der Belastungen
wie oben berechnet angenommen wird?

Gehen Sie dabei von st\"uckweise konstanten Dichtefunktionen aus,
\bdm
f(Q)=f_k^D, \quad \text{falls $Q$ in Klasse $k$}
\edm
(mit der Klassendichte $f_k^D$ gleich der relativen
Klassenh\"aufigkeit geteilt durch die Klassenbreite)
und berechnen Sie die Zeitverz\"ogerung pro Kilometer und Jahr mittels
\bdm
\Delta T\sub{tot}=n\int\limits_K^{\infty} (T(Q)-T_0) f(Q).
\diff{Q}
\edm
(was bedeutet in dieser Formel die Gr\"o\3e $n$?)

\ei

\subsection*{L\"osung (schematisch)}

\bi
\item[(a,b)] Wertetabelle:
\begin{center}
\begin{tabular}{|l|l|l|l|l|} \hline
Klassenindex $k$ & Wertebereich (Fz/h) & $f_k$ & $F_k$ & $f_k^D$ \\ \hline
1 & 0-200 & $\frac{150*6}{365*24}$ & $f_1$ & $\frac{f_1}{200}$ \\
2 & 200-300 & $\frac{100*6}{365*24}$ & $f_1+f_2$ & $\frac{f_2}{100}$ \\
...&...&...&...&...\\
8 & 1400-1600 & ...& 1 & $\frac{f_8}{200}$ \\ \hline
\end{tabular}
\end{center}
\item[(c)] Quantilswert $x_q$ (``$q$-Quantil'') wie in Statistik:
\bdm
x_q = x_{k'}^u
     + \frac{q-F_{k'-1}}{f_{k'}} \Delta x_{k'}
\edm
    mit $k'$ so dass $F_{k'-1}\le q$, aber $F_{k'} > q$

\item[(d)] 30 h: Anteil $1-q=30/(365*24)$ also $q=1-30/(365*24)$
(ungef\"ahr 99.6\%); 150 h analog: $q=0.983$.

\item[(e)]
Das Integral spezialisiert sich f\"ur $\Delta t$ in Stunden zu
\bdm
\Delta T\sub{tot}=\frac{5n}{60}\int\limits_{Q_{0.983}}^{1600} 
f(Q)\left(\frac{Q}{Q_{0.983}}-1\right) \diff{Q}.
\edm
mit $n$ der Gesamtzahl der Fahrzeuge pro Jahr; die 60 im Nenner kommt daher,
dass $T_0=1/60$ h pro km; 
$Q_{0.983}$ ist das zu 150 h entsprechende Quantil
(es sollte so um $Q_{0.983}=1380$ liegen). 

Auswertung des Integrals geht \"uber
die beiden obersten Klassen; es ergeben sich jeweils Integrale der
Form $\int (a+bQ)\diff{Q}=\frac{b}{2}Q^2+aQ+c$.
\ei


\end{document}
%########################################################

